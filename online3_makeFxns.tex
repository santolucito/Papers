\subsection{Component function generation}\label{makeFxns}
% makeFxns

Recalling the solution space of programs defined in Section \ref{problem}, \ourTool/ must now find fisrt order function for each of the higher order functions that are still candidates.
For a given higher order function, \ourTool/ can choose component functions by reusing the weighted type matching algorithm from Listing \ref{valueAlgo}.
We are looking for first order functions that will type check against the component signature.
In order to do this, we need to know the concrete type signature of higher order function when applied to the examples.

Similar to Listing \ref{hoRank}, we show an example of how type matching is applied over first order functions in Listing \ref{compRank}.

\begin{lstlisting}[caption=Ranking component function,label=comprank]
examples ::            [Int] -> [Int]
map ::   (a   -> b)   -> [a]   -> [b]
mapEx :: (Int -> Int) -> [Int] -> [Int]

component ::
      Int    -> Int
f1 :: a      -> b      -- value is 21
f2 :: Int    -> a      -- value is 31
f3 :: Int    -> Int    -- value is 41
f4 :: [Bool] -> [Bool] -- value is Nothing
\end{lstlisting}

\subsection{Initial Values}

In the case that neither of those situations hold, we try to apply a value to the first order function.
For instance, if the component signature is \codeinline{::Int->Int}, and we have the first order functions \codeinline{(+)::Int->Int->Int}, we apply some initial values to \codeinline{(+)} to get a new function (e.g. \codeinline{(+1)}) that fits the component signature.

If the initial value's type is an instance of Monoid, we can extract the unit value (named mempty in Haskell's monoid typeclass\cite{monoid}) to use as our initial value. For lists, the unit element is []. However, there are two valid monoids for numbers, using either (+) or (*) as the operators and resulting in unit elements 0 and 1 respectively. We take both of these values (along with other common, useful values of -1, and 2) as possibilities since the cost of testing both values is relatively small.

Additionally, requiring our users to write monoid instances for their datatypes may be a nuisance. However, users may have some domain knowledge that a particular value, or set of values, may be useful in their application. Since our system automatically considers functions defined in the user code base, users may simply write their own specializations of the higher order functions, or provide useful initial values, to be used in synthesis. 

\begin{lstlisting}[caption=adding default initial values]
-- to use 5 as an initial value for foldl
foldl :: (a -> b -> a) -> [b] -> a
foldl5 f i o = foldl f 5 i o

-- to use 5 as an initial value in all recursions
x :: Int
x = 5
\end{lstlisting}

Presented with the problem of finding integer values to satisfy the examples may initially seem like a good application for an SMT solver. However, keep in mind that we do not in general know what we are trying to solve - the actual use of these variables is hidden within the function definition. Since in this work we want to stick to a type directed approach, rather than code analysis, we will not be able to unravel these functions.

In the implementation, we actually build new functions with the name of the composed functions, and adjust the type signature accordingly.

Before proceeding to component function synthesis, we must also address an issue first presented in Section \ref{problem}.
It is possible for a higher order function to need initial values in addition to a component function.
For example, the \codeinline{map} function only takes a first order function, while \codeinline{foldl :: (a-> b-> a)-> a-> [b]-> a} requires an initial value for \codeinline{a}.
While the process described so far handles the former, initial values must also be addressed.

For each higher-order function, we supply it with arguments until it is compatible with the type signature implied by the example set. In accordance with our problem definition in Section \ref{problem}, we can assume the final argument of the function is the input. Any other initial value types are satisfied by selecting from a pool of default values, and function types are satisfied by searching for first-order functions that would make the resulting signatures match. If it is not possible to find values that fit, the search moves on to the next higher-order function.

To identify initial values in a type signature, we can use our previous assumption that all higher order function have been partially curried to the type \codeinline{_ -> *-> *}. Adding the further assumption that only one first order function maybe be passed to the higher order function, we simply tag any non-function type in the hole as an initial value.



%If the types of a higher order function do not match, we 
