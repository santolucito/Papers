With the synthesis engine constructed, the system is ready to synthesize programs from examples.
Multiple programming-by-example queries can then be answered using this engine.
The synthesis engine only needs to be reconstructed when there are new library imports, or when there is a revision of the user-supplied code.

When examples are provided, the synthesis engine finds a suitable refinement type for a hypothetical function that could fit that example.
Then, \ourTool/ filters and ranks the higher order functions based on the refinement types known to the engine and the example types provided.
Once the candidate higher functions are identified, \ourTool/ will select and build first order functions that match the type of the higher order function's component signature to build a final set of candidate programs.

Each of these candidate programs is executed in best-first order against the set of inputs.
Whenever a function produces the correct outputs for each input, it is said to fit, and is reported to the user.
This search continues until the space is exhausted or it is manually interrupted. 
The search will always terminate since we are working over a finite space of generated functions, are our type reductions are strictly decreasing, which we will explain in Section \ref{typeMatch}

% getExampleType
% assignRType
\subsection{Refinement types for examples}
% getExampleType
As in Section \ref{HORtypeInf}, we also consider two cases for examples. The first, where the example input and output types match up to the top level type constructor, and the the case where the types do not match.

% assignRType
In the case that the types do match, we find the set of refinement types that the examples satisfy. Generating refinement type predicates about the size of the input and output, as in Section \ref{HORtypeInf}, we apply the same algorithm from Listing \ref{listing:addRType}. 
For instance, an example set for \codeinline{filter (>3)} might look as follows:

\begin{lstlisting}[caption=Refinement type inference for examples,label=exRTypeGen]
ex :: [Int] :-> [Int]
ex = [[1,2,3] :-> [1,2,3],
      [1,3,4] :-> [1,3],
      [4,6,8] :-> []]
       
exRType ::
  inExs :[Int] :-> 
 {outExs:[Int] | len inExs <= len outExs }
\end{lstlisting}

\noindent and have the final refinement type of \codeinline{exRType}, since all of the examples suggest that the output list does not grow. 
Again, when the types do not match we assign the \codeinline{noRType} flag to the examples, as we did for higher order functions in Listing \ref{listing:addRType}.
We can now reduce our search space to only higher order functions with the same refinement type that matches the examples' refinement type. 


\subsection{Type match ranking}\label{typeMatch}

Once \ourTool/ has both the base and refinement types for the examples and higher order functions, it can can prune and order this set (line 17 of Listing \ref{listing:Algo}).
The first step is to simply filter the higher order function candidates over equality of refinement types.
Additionally, \ourTool/ will check the example types are concrete versions of the input/output types of the higher order function with the infix (for clarity) \codeinline{isConcreteTypeOf} function.
For type A to be a concrete version of type B, there must exist some type C (possibly equal to type B), such that both A and B can be instantiated to that type.
The above requirement is then that there is some way to unify these two types - a familiar problem\cite{typeUnif}.

\begin{lstlisting}[caption=Pruning based on types]
filter (exRType ==) higherOrderRTypes
filter (exType `isConcreteTypeOf') higherOrderComponentTypes
\end{lstlisting}

Once these higher order functions have been culled from the pool of candidates, we update their ranks that had been assigned in Section \ref{HORtypeInf} from code locality.
The higher order function can advance in the ranking by using a value function to find out exactly how much the example type \codeinline{isConcreteTypeOf} to the input/output types of candidate higher order function.

In Listing \ref{valueAlgo}, we present a demonstration of part of this ranking algorithm.
As we traverse the tree structure of the type, the more pieces of the type signature that match, the higher the value of that match. 
However, if there is a type constructor mismatch, the two types can never be reconciled, and the entire value gets nothing.

\begin{lstlisting}[caption=Type closeness ranking algorithm (sample),label=valueAlgo]
value :: Type -> Type -> Maybe Int
value (TyFun i1 o1) (TyFun i2 o2) =
   1 + (value i1 i2) + (value o1 o2)
value (TyCon n1) (TyCon n2) =
   if (n1==n2) then 20 else Nothing
value (TyCon n1) (TyVar _) = 10
value _ _ = Nothing
\end{lstlisting}

As an example of how this value function is applied the higher order functions, imagine we have three map functions specialized on particular values. 
The fully polymorphic map will score 1 point for having a function between input and out, 2 points for both having lists, and 20 points for a type variables matching a type constructor, for a total of 5 points. The mapI for Ints, will score the same, but score 20 points for each matching type constructors instead of 10 points for each type variable matched to a type constructor. The mapB for boolean value gets nothing since there is no way to reconcile that type to the example type.

\begin{lstlisting}[caption=Ranking higher order function,label=horank]
examples ::             [Int] :-> [Int]
map  :: (a    -> b)    -> [a]    -> [b]
mapI :: (Int  -> Int)  -> [Int]  -> [Int]
mapB :: (Bool -> Bool) -> [Bool] -> [Bool]

-- map  scores 5
-- mapI scores 43
-- mapB scores Nothing
\end{lstlisting}


\subsection{Component function generation}\label{makeFxns}
% makeFxns

With an ordered set of higher order functions, we must now find first order functions to act as the component function of the higher order function.
To choose the component function we reuse the weighted type matching algorithm from Listing \ref{valueAlgo}.
In order to do this, we need to know the type signature of higher order function when applied to the examples.
By specializing the higher order function on the example type, we now know the concrete type of the component function.
Since examples must be given as a concrete type, we can always specialize a our candidate higher order function. 
Similar to Listing \ref{hoRank}, we show an example of how type matching is applied over first order functions in Listing \ref{compRank}.

\begin{lstlisting}[caption=Ranking component function,label=comprank]
examples ::            [Int] -> [Int]
map ::   (a   -> b)   -> [a]   -> [b]
mapEx :: (Int -> Int) -> [Int] -> [Int]

component ::
      Int    -> Int
f1 :: a      -> b      -- value is 21
f2 :: Int    -> a      -- value is 31
f3 :: Int    -> Int    -- value is 41
f4 :: [Bool] -> [Bool] -- value is Nothing
\end{lstlisting}


Given this partially concrete type for the higher order function, we search for first order functions that will fit the component signature.
If the component signature is a concrete instance of the first order function, we can also accept it as a possibility.
This occurs when we try to use \codeinline{id::a->a} as a component for \codeinline{::Int->Int}.

Note that at this stage of synthesis, we can never have a first order function that is a concrete instance of the component signature. 
An example of this would be trying to use \codeinline{negate::Int->Int} as a component for \codeinline{::a->a}
Given our requirement that all type variables are determined by the uncurried input from Section \ref{problem}, we will have specialized all type variables with the \codeinline{specializeOn} function.
That means the component function must be concrete at this stage.
This is really good for performance.

A \textit{dismantling procedure} prunes the first order function search space by using information from the top level examples, and the current state of synthesis.

\subsection{Initial Values}
In the case that neither of those situations hold, we try to apply a value to the first order function.
For instance, if the component signature is \codeinline{::Int->Int}, and we have the first order functions \codeinline{(+)::Int->Int->Int}, we apply some initial values to \codeinline{(+)} to get a new function (e.g. \codeinline{(+1)}) that fits the component signature.

If the initial value's type is an instance of Monoid, we can extract the unit value (named mempty in Haskell's monoid typeclass\cite{monoid}) to use as our initial value. For lists, the unit element is []. However, there are two valid monoids for numbers, using either (+) or (*) as the operators and resulting in unit elements 0 and 1 respectively. We take both of these values (along with other common, useful values of -1, and 2) as possibilities since the cost of testing both values is relatively small.

Additionally, requiring our users to write monoid instances for their datatypes may be a nuisance. However, users may have some domain knowledge that a particular value, or set of values, may be useful in their application. Since our system automatically considers functions defined in the user code base, users may simply write their own specializations of the higher order functions, or provide useful initial values, to be used in synthesis. 

\begin{lstlisting}
-- to use 5 as an initial value for foldl
foldl :: (a -> b -> a) -> [b] -> a
foldl5 f i o = foldl f 5 i o

-- to use 5 as an initial value in all recursions
x :: Int
x = 5
\end{lstlisting}

Presented with the problem of finding integer values to satisfy the examples may initially seem like a good application for an SMT solver. However, keep in mind that we do not in general know what we are trying to solve - the actual use of these variables is hidden within the function definition. Since in this work we want to stick to a type directed approach, rather than code analysis, we will not be able to unravel these functions.

In the implementation, we actually build new functions with the name of the composed functions, and adjust the type signature accordingly.

Before proceeding to component function synthesis, we must also address an issue first presented in Section \ref{problem}.
It is possible for a higher order function to need initial values in addition to a component function.
For example, the \codeinline{map} function only takes a first order function, while \codeinline{foldl :: (a-> b-> a)-> a-> [b]-> a} requires an initial value for \codeinline{a}.
While the process described so far handles the former, initial values must also be addressed.

For each higher-order function, we supply it with arguments until it is compatible with the type signature implied by the example set. In accordance with our problem definition in Section \ref{problem}, we can assume the final argument of the function is the input. Any other initial value types are satisfied by selecting from a pool of default values, and function types are satisfied by searching for first-order functions that would make the resulting signatures match. If it is not possible to find values that fit, the search moves on to the next higher-order function.

To identify initial values in a type signature, we can use our previous assumption that all higher order function have been partially curried to the type \codeinline{_ -> *-> *}. Adding the further assumption that only one first order function maybe be passed to the higher order function, we simply tag any non-function type in the hole as an initial value.



%If the types of a higher order function do not match, we 


