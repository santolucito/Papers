%
% LaTeX template for prepartion of submissions to PLDI'16
%
% Requires temporary version of sigplanconf style file provided on
% PLDI'16 web site.
% 
\documentclass[pldi]{sigplanconf-pldi16}
% \documentclass[pldi-cameraready]{sigplanconf-pldi16}

%
% the following standard packages may be helpful, but are not required
%
\usepackage{SIunits}            % typset units correctly
\usepackage{courier}            % standard fixed width font
\usepackage[scaled]{helvet} % see www.ctan.org/get/macros/latex/required/psnfss/psnfss2e.pdf
\usepackage{url}                  % format URLs
\usepackage{listings}          % format code
\usepackage{enumitem}      % adjust spacing in enums
\usepackage[colorlinks=true,allcolors=blue,breaklinks,draft=false]{hyperref}   % hyperlinks, including DOIs and URLs in bibliography
% known bug: http://tex.stackexchange.com/questions/1522/pdfendlink-ended-up-in-different-nesting-level-than-pdfstartlink
\newcommand{\doi}[1]{doi:~\href{http://dx.doi.org/#1}{\Hurl{#1}}}   % print a hyperlinked DOI

\usepackage{comment}


\usepackage{minted}
\usepackage{tcolorbox}
\usepackage{etoolbox}
\BeforeBeginEnvironment{minted}{\begin{tcolorbox}}%
\AfterEndEnvironment{minted}{\end{tcolorbox}}%

\begin{document}

\title{Instructions for Submission to PLDI'16}

%
% any author declaration will be ignored  when using 'pldi' option (for double blind review)
%

\authorinfo{Person 1 \and Person 2}
{\makebox{A Department} \\
\makebox{A University}  \\
\makebox{A Place, AS 12345}}
{\{person1,person2\}@cs.auniv.edu}

\maketitle

\begin{abstract}
  We present programming by example that can utilize user defined higher order functions.
  We use refinement types to prune the search space of higher order functions.
  Since our refinement types can be under approximating, we can extend liquid haskell to support arbitraty syntax extensions.
  We introduce the concept of dismantling procedures to guide first order function synthesis.
  We use results from paramatricity as our dismantling procedure.
\end{abstract}

\section{Motivating Examples}

A core part of the functional programming experience is writing higher order functions. Many user's write higher order functions first, then combine them in interesting and useful ways. Library authors often provide users with many higher order functions to allow users more easily write their applications. With this in mind, we hope to provide a synthesis engine that is able to leverage user defined higher order functions. 

Since users write higher order functions with a deep understanding of the domain, using them will produce code that is more idiomatic and easier to understand then using generic higher order functions. Additionally, fewer examples are needed because we have access to the domain specific knowledge encoded by the user library.

\begin{minted}[fontsize=\footnotesize]{haskell}
--User has their own library of fxns
map2 :: (a -> b) -> [a] -> [b]
map2 f (x:xs) = f x : f x : map2 f xs

mapR :: (a -> a) -> [a] -> [a]
mapR f (x:xs) = mapR f xs ++ [f x]

--and wants to use it synthesize
exs :: [[Bool] ~> [Bool]]
exs = [[1, 2, 3] ~> [1, 1, 2, 2, 3, 3]]
\end{minted}


\section{System Flow}
First we build the synthesis engine by assigning refinement types to every higher order function in the user's library.
The list of refinement types is sorted by closeness to the user - user's functions first, explicit imports second, and standard library last.

With the engine built, we enter the synthesis stage when the user is ready to process some examples. 
We assign a refinement type to the examples and match that refinement type to the higher order functions.
The ranking of the higher order functions is adjusted based on how close the refinement types match, if at all.

Once candidate higher order functions have been chosen, we proceed to chose the first order, or component, function that fits the type the candidate higher order function.
With a set of possible programs (where a program is combination of higher order and first order functions), we apply programs to examples until we find one that satisfies all the examples.
Because we have kept a rank of best choices, the first correct one is also likely the most stylistic.

\section{Building the engine}

\subsection{Refinement type inference}
We first compile \textbf{map2} and \textbf{mapR} into the synthesis engine, which will later be used to find an implementation of \textbf{exs}.
Compiling the higher order functions into the synthesis engine is the equivalent of generating hypotheses about the higher order function in $\Lambda^2$.
We separate these hypotheses into two classes - \textit{search space reducers} and \textit{subexample generators}.
The search space reducers take the form of refinement types, while the subexample generators are functions.

Refinement types are chosen by generating and applying many examples with QuickCheck for each higher order function, as in \textbf{filterExs}.
We type check these examples against a pool of common and simple refinement types to chose a useful and correct refinement type for the higher order function.

\begin{minted}{haskell}
testInputs = [( (>20), [10,20,30]
              , even, [6,7,8]
              , True, ['a','b','c'])]
filterExs = map filter testInputs
filter  :: (a -> Bool)
        -> xs:[a]
        -> {ys:[a] | len xs >= len ys}
map2    :: (a -> b)
        -> xs:[a]
        -> {ys:[a] | len xs < len ys}
\end{minted}

The subexample generation functions are synthesized from examples collected by wrapping the component function in a state monad to collect a record of its executions.
We now have a set of examples for top-level input, top-level output, and subexample.
We can recursively call PBE... This isn't the right way I think.


\begin{minted}{haskell}
p' p x = write x >> write (p x) >> return (p x)
filterSubs = map (filter p') testInputs
{- ([10,20,30] ~> [30] ~> [10 ~> False, 20 ~> False, 30 ~> True]
   ,[6,7,8] ~> [6,8] ~> [6 ~> True, 7 ~> False, 8 ~> True]
   ,['a','b'] ~> ['a','b'] ~> ['a' ~> True, 'b' ~> True] -}
   
filterGen :: [[a] ~> [a]] -> [(a ~> Bool)]
filterGen = synthesize filterSubs
mapGen :: [[a] ~> [b]] -> [(a ~> b)]
\end{minted}


\subsection{Synthesis Stage}
Once the synthesis engine has been constructed, the system is ready to answer function fitting queries. When examples are provided, the synthesis engine finds a suitable refinement type for a hypothetical function that could fit that example. Then, this refinement type is matched via builtin type-checking against the higher order functions known to the engine.

Once the candidate functions are identified, a best-first search is initiated over combinations of the higher-order functions curried with each component function so that the combination type-checks. Each of these is executed against the set of inputs. Whenever a function produces the correct outputs for each input, it is said to fit, and is reported to the user. This search continues until the space is exhausted or it is manually interrupted.

As in section \ref{HORtypeInf}, we also consider two cases for examples. The first where the example input and output types match up to the top level type constructor, and the the case where the types do not match.

In the case that the types do match, we find the set of refinement types that the examples satisfy. Generating refinement type predicates about the size of the input and output, as in section \ref{HORTypes}, we run type inference on the input / output examples to \markk{ALEX, the following is wrong, what is a basic rtype anyway?} get a basic refinement type. Then, the input-output pairs are matched against a fixed set of predicates to enrich the refinement type. These predicates could impose constraints such as the input list being of equal, lesser, or greater size to the output list. \markk{end things being junk} For instance, the example set:
\begin{minted}{haskell}
exs = [[1,2,3,4] :-> [1,3],
       [2,4,6,8] :-> [],
       [5,7] :-> [5,7]]
\end{minted}
\noindent would have an inferred base type \markk{this isn't inferend right now, if we want to do that we would be using ghc, not LiqHask. check http://stackoverflow.com/questions/8963488/automatically-add-type-signatures-to-top-level-functions for a script to test} of \texttt{[Int] -> [Int]} and its final refinement type would be \texttt{xs:[Int] -> \{ ys:[Int] | len xs >= len ys \}}, since all of the examples suggest that the output list does not grow. The addition of these refinement type predicates dramatically reduces the search space in practice.

Once these higher order functions have been culled from the pool of candidates, we iterate through each in best-first order. For each higher-order function, we supply it with arguments until it is compatible with the type signature implied by the example set. By convention, we take the final argument of the function to be the input, which means that higher-order functions that take multiple inputs have to explicitly be uncurried. Value types are satisfied by selecting from a pool of default values, and function types are satisfied by searching for first-order functions that would make the resulting signatures match. If it is not possible to find values that fit, the search moves on to the next higher-order function.

\subsection{Dismantling procedures}

A \textit{dismantling procedure} prunes the first order function search space.
Subexample generation from that other paper\cite{isil} is one example of a (powerful) dismantling procedure. Subexample generation also gives the ability to recursively call the synthesis engine to generate programs with multiple applications of a high order function.
Because subexample generation is hard, and we dont know how to do it yet, we present a different idea.
Our dismantling procedure will deduce a specialized component signature given a higher order signature and the examples signature.

To choose the component function we use a sound, but not complete subtyping.
When two types have this relations, we will say type a \textit{generalizes} type b.
\markk{someone probably wrote something about this, just need to find out where and cite.}
If we have the higher order function map :: (a$\to$b) $\to$ [a] $\to$ [b] and the examples :: [Int] $\to$ [Int], then our component function f might have f::a$\to$b, or f::Int$\to$Int, or f::Int$\to$a, but certainly not f::[Bool]$\to$[Bool].

\begin{minted}[fontsize=\footnotesize]{haskell}
map :: (a -> b) -> [a] -> [a]
exs :: [Int] :-> [Int]

--component function must generalize
f :: Int -> Int

goodFxn1 :: Int  -> Int
goodFxn2 :: Int  -> a
goodFxn3 :: a    -> a
badFxn   :: Bool -> Bool
\end{minted}

\subsection{Handling folds}
We identify two separate classes of higher order functions - those that take a single first order function, and those that need initial values in addition to a function. The \texttt{map} function only takes a first order function, while \texttt{foldl :: $(a\to b\to a)\to a\to [b]\to a$} requires an initial value for \texttt{a}. While the process described so far handles the former, initial values must also be addressed.

To identify initial values in a type signature, we can use our previous assumption that all hihgher order function have been partially curried to the type \texttt{$\_\to *\to*$}. Adding the further assumption that only one first order function maybe be passed to the higher order function, we simply tag any non-function type in the hole as an initial value. 

Since examples must be given as a concrete type, we can always specialize a our candidate higher order function. If the initial type is an instance of Monoid, we can extract the unit value (mempty in the moiod typeclass) to use as our initial value. 

For lists, the unit element is []. However, there are two valid monoids for numbers, using either (+) or (*) as the operators and resulting in unit elements 0 and 1 respectively. We take both of these values as possibilities since the cost is small \markk{compared to all the refinement type stuff we are doing.}

Requiring our users to write monoid instances may be a nuisance. Additionally, users may have some domain knowledge that a particular value may be useful in their applications of folds. If this is the case, users may write specializations of these functions to be used in synthesis. Since our system automatically considers functions defined in the user code base, \markk{finish sentence}.

\begin{minted}[fontsize=\footnotesize]{haskell}
--to use 5 as an initial value for foldl
foldl :: (a -> b -> a) -> [b] -> a
foldl5 f i o = foldl f 5 i o
\end{minted}


\subsection{Soundness and Completeness}
\markk{for completness, can we show that we are complete within a specific domain? Also see \ref{extLiqHask} for more on completness.}

From the procedure outlined above, it is clear that no function will be returned by the algorithm that does not fit the examples given, since functions are validated before being reported. Still, it is possible for the synthesis procedure to return a function that does not capture the user's intent. Generally, this can be solved by supplying more examples to narrow the set of possible fitting functions.

However, depending on what the user is trying to synthesize, and which examples have been provided, it is possible for new examples to increase the search space. If, for example, a user gives only positive examples for a \texttt{filter}, the refinement type predicate discovery will assume that the lists do not change size, and will likely return \texttt{map id} as a result.

On the other hand, the set of functions that the algorithm can produce is fairly broad. It is able to search through the entire space of higher order functions that have been specialized with a first-order function, when considering the functions that are in scope. We will see in section~\ref{sec:evaluation} how broad this space actually is.




\begin{comment}
\section{Introduction}
We enable programming by example that leverages user defined higher order functions.
There are two steps, preprocessing and user-level synthesis, which could be compared to compile time and runtime for the PBE engine.

In the runtime process, we will choose an appropriate candidate higher order function.
With that higher order function, we will need to synthesize its component function.
We then recursively apply out PBE engine.
This exactly the $\Lambda^2$ algorithm I think.

In the preprocessing step we synthesize refinement types for the user defined higher order functions to be used for determining which functions to apply to a given example set.
In order to synthesize the component functions at runtime, we will synthesize a subexample generation function in preprocessing.
We need to generate subexamples to be able to recursively apply the PBE engine.
In a fascinating and twisted recursion, the subexample generation function is also synthesized using the PBE engine. 

After the preprocessing step, from which we have obtained refinement types and subexample generation functions, we then move to the actual user-level synthesis.
We will generate refinement types for the given examples then use type checking to choose a good higher order function to explore.

\section{Preprocessing}
In preprocessing, we are generating hypotheses for all user defined higher order functions.
The hypotheses say something about the semantic meaning of hoe the function works.
We have refinement types which act as statements about the applicability of the function to example sets.
We have subexample generation functions for each higher order function that are statements about how the component function will behave.

\subsection{Refinement type generation}
Pull from a pool of template refinement types that fit the typeclass constraints. 
If we wanted to synthesize these, we probably could (though I'm not sure how or if it is useful at all).

\subsection{Subexample generation}
After choosing a higher order function to explore as a candidate for the user provided examples, will will need to also synthesize the component function.
For example if we have \textbf{map f list}, we need to also synthesize \textbf{f}.
We will rely on recursively calling our PBE engine on \textbf{f}, which means we need to have examples for \textbf{f}.
These examples can be extracted from the top level user provided examples, but that requires a subexample generation function for every higher order function.

We can generate as many examples for a higher order function as we want (using quickcheck's co/arbitrary classes).
We will also wrap the component function in a state monad that allows us to record its execution on all of the examples.

As an example, consider \textbf{map}.
Given the type of \textbf{map :: (a -> b) -> [a] -> [b]}, the subexample generation function will have type \textbf{:: [a] -> [b] -> [(a,b)]}.
We might generate the example input \textbf{(+1) [1,2,3]}.
We wrap the function \textbf{(+1)} in a monad so that everytime it is executed, we keep a record of its inputs and outputs, so we know that the component function mapped \textbf{[(1,2),(2,3),(3,4)]} to give an output from map of \textbf{[2,3,4]}.

Well now we have generated an example for the subexample generation function \textbf{example\_input = ([1,2,3],[2,3,4] ; example\_output = [(1,2),(2,3),(3,4)]}.
So, what if we try to run PBE on this? 
Interestingly, this is a slightly easier synthesis problem (than user-level synth) since we have the ability to generate as many examples as we want. 
There is also not as much of a time constraint since this is a preprocessing step, or "compile time cost", before the user starts actually running synthesis themselves. 
The other crazy thing is that we can apply the \textit{incompletely built} PBE engine recursively here, until we reach a synthesis problem that is easy to solve, then propagate that solution back up to the top level.

\section{User-level Synthesis}
We synthesizing the refinement types for examples we need to be careful to not generate the strongest possible refineent type.
Since example are inherently an underspecification, this can lead to problems.
As an example consider \textbf{filter}, for which all the examples a user provide may actually filter an element.
The strongest refinement type for this would  \textbf{xs:[a] -> {v:[a] | (len v) < (len xs)}} , while the refinement type for filter is correctly \textbf{filter :: (a -> Bool) -> xs:[a] -> {v:[a] | (len v) <= (len xs)}}.
We may need to deal with subtyping, but if we choose a limited and sensible pool of template refinement types to start with, we might be able to avoid that whole can of worms.  

\end{comment}

\bibliographystyle{abbrvnat}
\bibliography{sample}


\end{document}
