%
% LaTeX template for prepartion of submissions to PLDI'16
%
% Requires temporary version of sigplanconf style file provided on
% PLDI'16 web site.
% 
\documentclass[pldi]{sigplanconf-pldi16}
% \documentclass[pldi-cameraready]{sigplanconf-pldi16}

%
% the following standard packages may be helpful, but are not required
%
\usepackage{SIunits}            % typset units correctly
\usepackage{courier}            % standard fixed width font
\usepackage[scaled]{helvet} % see www.ctan.org/get/macros/latex/required/psnfss/psnfss2e.pdf
\usepackage{url}                  % format URLs
\usepackage{listings}          % format code
\usepackage{enumitem}      % adjust spacing in enums
\usepackage[colorlinks=true,allcolors=blue,breaklinks,draft=false]{hyperref}   % hyperlinks, including DOIs and URLs in bibliography
% known bug: http://tex.stackexchange.com/questions/1522/pdfendlink-ended-up-in-different-nesting-level-than-pdfstartlink
\newcommand{\doi}[1]{doi:~\href{http://dx.doi.org/#1}{\Hurl{#1}}}   % print a hyperlinked DOI


\usepackage[usenames,dvipsnames]{color}

\newcommand{\ruzica}[1]{\textcolor{Magenta}{\textsf{RP}: #1}}
\newcommand{\markk}[1]{\textcolor{Blue}{\textsf{MS}: #1}}
\newcommand{\alex}[1]{\textcolor{Orange}{\textsf{AR}: #1}}


\def\ourTool/{our tool}
\def\lhask/{liquidHaskell}

\newcommand{\codeinline}[1]{\lstinline[basicstyle=\small]{#1}}



\usepackage{comment}

\usepackage{graphicx}

\definecolor{identifierColor}{rgb}{0.65,0.16,0.16}
\definecolor{comment_color}{rgb}{0.50,0.66,0.4}
\definecolor{num_color}{gray}{0.55}

\lstset{
  basicstyle=\footnotesize,
  breaklines=true,
  frame=bottomline,
  language=haskell,
  %identifierstyle=\color{identifierColor},
  morecomment=[l][\color{comment_color}\ttfamily]{--},
  backgroundcolor=\color{white},   % choose the background color; you must add \usepackage{color} or \usepackage{xcolor}
  breakatwhitespace=false,         % sets if automatic breaks should only happen at whitespace
  %captionpos=b,                    % sets the caption-position to bottom
  %commentstyle=\color{mygreen},    % comment style
  %frame=single,	                   % adds a frame around the code
  keepspaces=true,                 % keeps spaces in text, useful for keeping indentation of code (possibly needs columns=flexible)
  keywordstyle=\color{blue},       % keyword style
  otherkeywords={*,let, Server, Replication, FaultGraph, rankRCG, print, fialProb, goal, ...},           % if you want to add more keywords to the set
  numbers=left,                    % where to put the line-numbers; possible values are (none, left, right)
  numbersep=5pt,                   % how far the line-numbers are from the code
  numberstyle=\tiny\color{num_color}, % the style that is used for the line-numbers
  rulecolor=\color{black},         % if not set, the frame-color may be changed on line-breaks within not-black text (e.g. comments (green here))
  showtabs=false,                  % show tabs within strings adding particular underscores
  stepnumber=1,                    % the step between two line-numbers. If it's 1, each line will be numbered
  stringstyle=\color{mymauve},     % string literal style
  %title=\lstname                  % show the filename of files included with \lstinputlisting; also try caption instead of title
  mathescape=true,
  tabsize=3,
  literate=*{->}{{\textcolor{blue}{$\to$}}}{1}
           {<-}{{\textcolor{blue}{$\leftarrow$}}}{1}
}
  
  
%\usepackage{minted}
%\usepackage{tcolorbox}
%\usepackage{etoolbox}
%\BeforeBeginEnvironment{minted}{\begin{tcolorbox}}%
%\AfterEndEnvironment{minted}{\end{tcolorbox}}%

\begin{document}

\title{Instructions for Submission to PLDI'16}

%
% any author declaration will be ignored  when using 'pldi' option (for double blind review)
%

\authorinfo{Person 1 \and Person 2}
{\makebox{A Department} \\
\makebox{A University}  \\
\makebox{A Place, AS 12345}}
{\{person1,person2\}@cs.auniv.edu}



\maketitle

\begin{abstract}
We present a new programming by example technique that efficiently synthesizes fitting functions that mix user-defined higher order functions with standard and third-party library code.

We present programming by example that can utilize user defined higher order functions through a dismantling procedure. We use refinement types to prune the search space of higher order functions. Since our refinement types can be under approximating, we can extend liquid Haskell to support arbitrary syntax extensions and maintain soundness. For component/first order functions, we use type directed synthesis. We might use results from paramatricity to synth first order fxns, but probably not.
\end{abstract}



(1) We use refinement type based inductive generalization (inductively deriving properties from examples).

(2) from the higher order examples we use deductive reasoning to automate refinement type inference on higher order function.

(3) We combine these two with abductive reasoning to create a ranking and perform a best-first enumerative search based on type closeness and code locality. 


As an introduction to \ourTool/, imagine that a user wants to synthesis the simple \codeinline{stutter} function that will duplicate each element of a list.
The user will provide an example, and \ourTool/ will synthesis a program \codeinline{concatMap (replicate 2)} that fit that example.

\begin{lstlisting}
exs :: [[Int] :-> [Int]]
exs = [[1, 2, 3] :-> [1, 1, 2, 2, 3, 3]]
\end{lstlisting}

We might also imagine that the user was able to complete a piece of this program by writing a function \codeinline{dupl} to duplicate an element. Now \ourTool/ will provide the solution \codeinline{concatMap dupl}, as well as the solution above.

\begin{lstlisting}
dupl :: a -> [a]
dupl x = [x,x]
\end{lstlisting}

Because \ourTool/ only searches for natural and idiomatic programs that use higher order functions, very few examples are needed. In this case the user wants a function that will take numbers from a list as long as the numbers are odd. Only a single example is needed for \ourTool/ to unambiguously find the function \codeinline{takeWhile odd}. Another valid function might be \codeinline{head} to take the first element. Searching for first order functions is an active research direction, but in this work we instead focus only on higher order functions. A discussion of integrating our tool with first order searching techniques in provided in Section \ref{evaluation}.

\begin{lstlisting}
exs :: [[Int] :-> [Int]]
exs = [[1, 2, 3] :-> [1]]
\end{lstlisting}

Working on user defined datatypes is also a commonplace task \ourTool/ supports. In the next example the user has provided a binary tree data structure and a function to map over it. For the sack of brevity, we show the synthesis the exceedingly simple program \codeinline{mapBTree not}.

\begin{lstlisting}
data BTree a = Nil |
               Branch (BTree a) a (BTree a)

mapBTree :: (a -> a) -> BTree a -> BTree 
mapBTree f Nil = Nil
mapBTree f (Branch b1 v b2) = 
  Branch (mapBTree f b1) (f v) (mapBTree f b2)

exs :: [BTree Bool :-> BTree Bool]
exs = [Branch Nil True Nil :->
       Branch Nil False Nil]
\end{lstlisting}

It may seem that if a user can write a higher order functions over custom data structures, they would not have a need to synthesize such functions.
However, imagine the incredibly common case of a user importing libraries.
Haskell's module system and large repository of libraries like Hackage and Stackage are an indispensable part of the language\cite{hackage,stackage}.
Often, a user is importing a library that is large, unfamiliar, and/or poorly documented.
Using \ourTool/, the user no longer needs an intimate knowledge of the library to makes use of the functions and datatypes, and can instead synthesize functions from examples.

As an example, we show code to transpose a music value from the Euterpea DSL for music\cite{Euterpea}.
Among other things, Euterpea defines a tree-like datatype called \codeinline{Music} and various functions for manipulating these types.
The user only needs to express the basic datatype as examples, and \ourTool/ can synthesize the \codeinline{solution} program.
The solution utilizes the functions from Euterpea; \codeinline{mMap} for mapping over music values, and \codeinline{(trans::Int->Music Pitch->Music Pitch)} to transpose a Music Pitch by a value.
Because we have synthesized a natural looking program, the user does not need to understand details of the library's function and data structures to be able to immediately gain an intuition about how the solution program works.

\begin{lstlisting}
import Euterpea

exs :: [Music Pitch :-> Music Pitch]
exs = [
  (Prim (Note qn (C,4)):+:Prim (Note qn (D,4)) :->
  (Prim (Note qn (D,4)):+:Prim (Note qn (E,4)) ]
        
solution = mMap (trans 2)
\end{lstlisting}

\markk{should we list the solution program that lambda squared would give? It would be long and full of cases and folds and maps}

In this section we formally define the space of functions we are interested in synthesizing. Using this definition we will show in section \ref{sound}, that although our algorithm is not complete in general (by the inherent nature of examples), it is complete for the subset of the language we define here.

We are interested in synthesizing higher order functions that manipulate data structures. We support the classic functions like \texttt{map, filter, foldl}, but also user defined higher order functions from user code, or imported modules. 

\begin{lstlisting}
solutionProgram ::
       (* -> types)  -- Component Function
    ->  types        -- Initial Values
    ->  *          -- Input
    ->  *          -- Output

types = * | * -> types

-- * matches on Type Variables and Constructors
\end{lstlisting}

We do not explicitly address synthesizing first order functions to fit the examples. We do employ a method to synthesize first order functions to act as the component functions for higher-order functions. Existing work has shown promising advances in synthesizing top level, first order functions\cite{potential, reviewers}. While it is out of the scope of this paper to go into details, we briefly discuss integration of dedicated first order synthesis procedures into our tool in section \ref{conclusions}.


\subsection{Example Syntax}
The user must give examples as a pair of values. We use the \texttt{:->} operator for clarity to differentiate between tuples and examples.
We require all higher order functions be be of a unified signature \texttt{$\_ \to * \to *$}, where the penultimate kind of the signature is the input and the final kind is the the output. \markk{Put in a note about data kinds here.} 

Functionally, this means we require that the user partially uncurry any higher order function they are interested in using during synthesis. This is a simple procedure, but requires user knowledge of which parameters to the function will be given by the examples. As an example consider 

\begin{lstlisting}
zipWith' :: (a -> b -> c) -> ([a],[b]) -> [c]
zipWith' f (xs,ys) = zipWith f xs ys
\end{lstlisting}

Any types that are between the input and first order function will be assumed to be initial values for the recursions. For example, \texttt{foldl (+)} needs an initial value of 0 to become the sum function.

%The liquidHaskell predicate applied to this signature will be of the effect of \texttt{len([a],[b]) = len([c])}.

\section{System Overview}



\begin{figure}[t]
  \centering
% Define block styles
\tikzstyle{block} = [rectangle, draw, fill=none, 
    text centered, sharp corners, minimum height=3em]
\tikzstyle{line} = [draw, -latex']
    
\begin{tikzpicture}[node distance = 7em, auto]
    % Place nodes
    \node [block](hofu) {select HO functions};
    \node [block, above left of=hofu] (libraries) {standard libraries};
    \node [block, above right of=hofu] (API) {user definded functions};
    \node [block, below of=hofu, node distance=5em] (refty) {assigning refinement types};
    \node [block, below of=refty, node distance=5em] (engine) {synthesis engine};
    \node [block, left of=engine, node distance=7em] (examples) {examples};
    \node [block, right of=engine, node distance=7em] (program) {program};

    
    % Draw edges
    \path [line] (libraries) -- (hofu);
    \path [line] (API) -- (hofu);
    \path [line] (hofu) -- (refty);
    \path [line] (refty) -- (engine);
    \path [line] (examples) -- (engine);
    \path [line] (engine) -- (program);
\end{tikzpicture}
  \caption{High-level structure of the algorithm.}
  \label{fig:high_level_overview}
\end{figure}

Figure~\ref{fig:high_level_overview} gives a high-level description of ways in which the components of our algorithm interact. Broadly speaking, there are two main stages in the algorithm. The offline (preprocessing) phase gathers the higher order declarations visible in the APIs and user-provided code, and assigns refinement types to them to build a custom synthesis \textit{engine}. This engine is then used during the online phase of the algorithm to search for functions that fit a set of supplied examples.

During the offline phase, the algorithm first scans the user-provided code, the libraries it imports, and the standard library to gather all of the functions and global values visible to the program. Then, it selects the higher-order functions from the set of all functions and values, and uses Liquid Haskell \cite{DBLP:conf/haskell/VazouSJ14, DBLP:conf/esop/VazouRJ13, DBLP:conf/icfp/VazouSJVJ14} to assign refinement types to each one. Finally, each higher-order function is assigned a weight based on locality\cite{DBLP:conf/pldi/GveroKKP13}. User-defined functions are given the highest priority, while direct imports are given less, and the standard libraries are given the least. Together with the first-order functions and values, these triples of higher-order functions with their refinement types and weights are collected to produce a synthesis engine.

Once this stage is complete, the user can examples to the synthesis engine, which will search the space of constructable functions for those that fit the examples. First, the engine computes a refinement type that fits the examples. This type is matched against the refinement types of the known higher-order functions, and the weights of each known function are adjusted based on how close the types match, if at all.

Once the candidate higher order functions have been chosen, the synthesis engine performs a best-first search for a program that fits all of the input and output examples by composing the candidates with first-order functions. For example, the higher-order function \texttt{map} might be supplied the \texttt{length} function if the example inputs are lists of lists of integers and the output examples are all lists of integers. The programs that are examined during the search are evaluated against the example set and are reported to the user as they match. Because the weights favor local declarations, the highest-ranked programs are likely to be the most idiomatic.

\markk{explain each of these lines in the rest of the paper, citing line numbers. once all of the lines are covered here we are done}
 We will explain each line of this in the proceeding Sections.
 
\begin{lstlisting}[caption=A pseudocode representation of the build and synthesis stages of the synthesis algorithm, label=listing:Algo]
main = do
  eng <- build
  ex  <- getExamples
  synth eng ex
  
build = do
  allTypes   <- collectTypesAndWeights
  allHOTypes <- filter isHigherOrder allTypes
  allRTypes  <- assignRTypes allHOTypes
  return (allTypes,allRTypes)
  
synth eng ex = do
  -- assign refinement types to examples
  exType   <- getExampleType ex
  exRType  <- assignRTypes exType
  -- make candidate functions and programs
  hoFxns <- rankByTypeMatch exRType eng
  progs  <- makeFxns exType hoFxns
  -- test the ranked list of possible programs
  validProgs <- filter (testOn ex) progs
\end{lstlisting}


\section{Offline: Synthesis Engine Construction}
\subsubsection{Refinement Type Generation}

%>  filter ishigherOrder tys
We first collect all of the type signatures from our sources (user code, imports, and standard library). We filter through these to select only the higher order functions. Because in Haskell the function type constructor (->) is right binding, any higher order functions must have parenthesis in the type signature, which provides a convenient filtering predicate.

%> let uHOTyps = f 3000 typSigs
%> let iHOTyps = f 2000 importSigs
%> let pHOTyps = f 1000 preludeTypSigs
In order to rank the higher-order functions, we assign weights based on their source location. User-defined functions are given the highest priority, while direct imports are given less, and the standard libraries are given the least. These rankings will contribute to the final ranking of candidate functions in the synthesis stage when we match the component function signatures on the examples.

%> hoRTyps <- mapM (addRType fc) (map fst allHOTyps)
%> HigherOrderFxn -> (injectRFxnType fxn .fst)
We automatically generate refinement types for our higher order functions that will be used to prune the search space in synthesis.
In the case that the input and output types of the higher order functions are the same (up to the top level type constructor), we generate a LiquidHaskell predicate that relates the size of those types.
the size relation that applies to the higher order functions must also apply to the examples in order to consider that function as a candidate.   

In the case that the input and output type are different, we note that the size measures between two different type constructors are not guaranteed to have any significance.
A relation on these values may be useful on occasion, but in practice is often only a confounding factor.
Since LiquidHaskell is the largest cost to our system, removing refinement type inference in these ambiguous cases provides a large performance gain.
In processing the Haskell standard library, base:Prelude, we remove \markk{FILL ME IN} cases of refinement type checking.
Hypothesis generation about these higher order functions is then deferred to the synthesis stage and where we use a subtype ranking system to good effect.

In order to derive these refinement types, we require all higher order functions be be of a unified signature \texttt{$\_ \to * \to *$}, where the penultimate kind of the signature is the input and the final kind is the the output. \markk{Put in a note about data kinds here.} Functionally, this means we require that the user partially uncurry any higher order function they are interested in using during synthesis.

This is a simple procedure, but requires user knowledge of which parameters to the function will be given by the examples. 
As an example consider 

\begin{lstlisting}
zipWith' :: (a -> b -> c) -> ([a],[b]) -> [c]
zipWith' f (xs,ys) = zipWith f xs ys
\end{lstlisting}
%The liquidHaskell predicate applied to this signature will be of the effect of \texttt{len([a],[b]) = len([c])}.


%> map rTypeTemplate ["=","<=",">="]
When the input and output types are the same, or use the same top level type constructor, we generate hypotheses as liquidHaskell predicates.
Our predicates specify size constraints on input and output of $\leq,\geq,=$.
For every predicate provided, we are able to more accurately prune the search space of higher-order functions, but we must test every higher-order function in scope on these predicates. 
Therefore, it is best to only select as many refinement types as is needed.
Although this offline stage only needs to be run once given a set of code and imports, liquidHaskell type checking is still fairly expensive.



\subsubsection{User defined data types}
%We focus only higher order functions that manipulate data structures
In order to support user defined data structures, we only require that a user implements some kind of measure\cite{realWorldLiquid} over their data structure.
This size function will help the system determine size constraints on the examples, so that we can pick higher order functions that might actually work.
In fact, a size function could just be a constant function, which means the system will test every higher-order function that fits the types. 
\markk{Maybe this should even be a builtin default? If liquid haskell says no def for len, just set the type to true - thats an easy hack}

As an example, take the code from section \ref{examples} for synthesizing a music function.
the user would have needed to provide a measure function for Music a.
This measure will allow liquidHaskell to draw conclusions about the size of examples of type [Music a :$\to$ Music a], as well as conclusions about higher order functions over the Music data structure.

\begin{lstlisting}
import Euterpea

{-@ measure len @-}
len :: Music a -> Int
len m =
  case m of
    Prim _  -> 1
    m1 :+: m2 -> len m1 + len m2
    m1 :=: m2 -> len m1 + len m2
    Modify c m -> len m
\end{lstlisting}





\section{Online: Fitting Functions to Examples} \label{synth}
With the synthesis engine constructed, the system is ready to synthesize programs from examples.
Multiple programming-by-example queries can then be answered using this engine.
The synthesis engine only needs to be reconstructed when there are new library imports, or when there is a revision of the user-supplied code.

When examples are provided, the synthesis engine finds a suitable refinement type for a hypothetical function that could fit that example.
Then, \ourTool/ filters and ranks the higher order functions based on the refinement types known to the engine and the example types provided.
Once the candidate higher functions are identified, \ourTool/ will select and build first order functions that match the type of the higher order function's component signature to build a final set of candidate programs.

Each of these candidate programs is executed in best-first order against the set of inputs.
Whenever a function produces the correct outputs for each input, it is said to fit, and is reported to the user.
This search continues until the space is exhausted or it is manually interrupted. 
The search will always terminate since we are working over a finite space of generated functions, are our type reductions are strictly decreasing, which we will explain in Section \ref{typeMatch}

% getExampleType
% assignRType
\subsection{Refinement types for examples}
% getExampleType
As in Section \ref{HORtypeInf}, we also consider two cases for examples. The first, where the example input and output types match up to the top level type constructor, and the the case where the types do not match.

% assignRType
In the case that the types do match, we find the set of refinement types that the examples satisfy. Generating refinement type predicates about the size of the input and output, as in Section \ref{HORtypeInf}, we apply the same algorithm from Listing \ref{listing:addRType}. 
For instance, an example set for \codeinline{filter (>3)} might look as follows:

\begin{lstlisting}[caption=Refinement type inference for examples,label=exRTypeGen]
ex :: [Int] :-> [Int]
ex = [[1,2,3] :-> [1,2,3],
      [1,3,4] :-> [1,3],
      [4,6,8] :-> []]
       
exRType ::
  inExs :[Int] :-> 
 {outExs:[Int] | len inExs <= len outExs }
\end{lstlisting}

\noindent and have the final refinement type of \codeinline{exRType}, since all of the examples suggest that the output list does not grow. 
Again, when the types do not match we assign the \codeinline{noRType} flag to the examples, as we did for higher order functions in Listing \ref{listing:addRType}.
We can now reduce our search space to only higher order functions with the same refinement type that matches the examples' refinement type. 


\subsection{Type match ranking}\label{typeMatch}

Once both the examples and higher order functions have refinement types, we can prune this set over equality.
The example types also must be concrete versions of the input/output types of the higher order function.
For type A to be a concrete version of type B, there must exist some type C (possibly equal to type B), such that both A and B can be instantiated to that type.

\begin{lstlisting}[caption=Pruning based on types]
filter (exRType ==) higherOrderRTypes
filter (exType `isConcreteTypeOf') higherOrderComponentTypes
\end{lstlisting}

Once these higher order functions have been culled from the pool of candidates, we update their ranks that had been assigned in Section \ref{HORtypeInf} from code locality.
The higher order function can advance in the ranking by using a value function to find out exactly how much the example type \codeinline{isConcreteTypeOf} to the input/output types of candidate higher order function.

In Listing \ref{valueAlgo}, we present a demonstration of part of this ranking algorithm.
As we traverse the tree structure of the type, the more pieces of the type signature that match, the higher the value of that match. 
However, if there is a type constructor mismatch, the two types can never be reconciled, and the entire value gets nothing.

\begin{lstlisting}[caption=Type closeness ranking algorithm (sample),label=valueAlgo]
value :: Type -> Type -> Maybe Int
value (TyFun i1 o1) (TyFun i2 o2) =
   1 + (value i1 i2) + (value o1 o2)
value (TyCon n1) (TyCon n2) =
   if (n1==n2) then 20 else Nothing
value (TyCon n1) (TyVar _) = 10
value _ _ = Nothing
\end{lstlisting}

As an example of how this value function is applied the higher order functions, imagine we have three map functions specialized on particular values. 
The fully polymorphic map will score 1 point for having a function between input and out, 2 points for both having lists, and 20 points for a type variables matching a type constructor, for a total of 5 points. The mapI for Ints, will score the same, but score 20 points for each matching type constructors instead of 10 points for each type variable matched to a type constructor. The mapB for boolean value gets nothing since there is no way to reconcile that type to the example type.

\begin{lstlisting}[caption=Ranking higher order function,label=horank]
examples ::             [Int] :-> [Int]
map  :: (a    -> b)    -> [a]    -> [b]
mapI :: (Int  -> Int)  -> [Int]  -> [Int]
mapB :: (Bool -> Bool) -> [Bool] -> [Bool]

-- map  scores 5
-- mapI scores 43
-- mapB scores Nothing
\end{lstlisting}


\subsection{Component function generation}\label{makeFxns}
% makeFxns

Recalling the solution space of programs defined in Section \ref{problem}, \ourTool/ must now find first order function for each of the higher order functions that are still candidates (line 18 of Listing \ref{listing:Algo}).
For a given higher order function, \ourTool/ can choose component functions by reusing the weighted type matching algorithm from Listing \ref{valueAlgo}.
Since examples must be given as a concrete type, we can always partially specialize our candidate higher order function. 
We then search for first order functions that will type check against the partially specialized component signature.
This partial specialization is a way of extracting more information out of our examples, and significantly reduces the space of candidate first order functions.
Similar to Listing \ref{hoRank}, we show an example of how type matching is applied over first order functions in Listing \ref{comprank}.

\begin{lstlisting}[caption=Ranking component function,label=comprank]
examples ::            [Int] -> [Int]
map ::   (a   -> b)   -> [a]   -> [b]
mapEx :: (Int -> Int) -> [Int] -> [Int]

component ::
      Int    -> Int
f1 :: a      -> b      -- value is 21
f2 :: Int    -> a      -- value is 31
f3 :: Int    -> Int    -- value is 41
f4 :: [Bool] -> [Bool] -- value is Nothing
\end{lstlisting}

\subsection{Initial Values}

In addition to finding first order functions where the arity of the kinds is equal to the component function, we may also want ``larger'' functions that have been applied to initial values.
For examples, if the component signature is \codeinline{::Int->Int}, we may have the first order functions \codeinline{(+)::Int->Int->Int} in scope.
By applying some initial values to \codeinline{(+)}, we can get a new function (e.g. \codeinline{(+1)::Int->Int}) that fits the component signature.

If the initial value's type is an instance of Monoid, we can extract the unit value (named mempty in Haskell's monoid typeclass\cite{monoid}) to use as our initial value. For lists, the unit element is []. However, there are two valid monoids for numbers, using either (+) or (*) as the operators and resulting in unit elements 0 and 1 respectively. We take both of these values (along with other common, useful values of -1, and 2) as possibilities since the cost of testing both values is relatively small.

Additionally, requiring our users to write monoid instances for their datatypes may be a nuisance. However, users may have some domain knowledge that a particular value, or set of values, may be useful in their application. Since our system automatically considers functions defined in the user code base, users may simply write their own specializations of the higher order functions, or provide useful initial values, to be used in synthesis. 

\begin{lstlisting}[caption=adding default initial values]
-- to use 5 as an initial value for foldl
foldl :: (a -> b -> a) -> [b] -> a
foldl5 f i o = foldl f 5 i o

-- to use 5 as an initial value in all recursions
x :: Int
x = 5
\end{lstlisting}

Presented with the problem of finding integer values to satisfy the examples may initially seem like a good application for an SMT solver.
However, keep in mind that we do not in general know what we are trying to solve - the actual use of these variables is hidden within the function definition. Since in this work we maintain a primarily type directed approach, rather than code analysis, we will not be able to unravel these functions.

We must also address the issue first presented in Section \ref{problem}, that it is possible for a higher order function to need initial values in addition to a component function.
For example, the \codeinline{map} function only takes a first order function, while \codeinline{foldl :: (a-> b-> a)-> a-> [b]-> a} requires an initial value for \codeinline{a}.
Using a similar process as for first order function application, we can apply values until the higher order function only needs the example input to complete execution.
To identify initial values in a higher order type signature, we can use our previous assumption that all higher order function have been partially curried to the type \codeinline{_ -> *-> *}. 
Adding the further assumption that only one first order function maybe be passed to the higher order function, we simply tag any non-function type in the hole as an initial value.

%In the implementation, we actually build new functions with the name of the composed functions, and adjust the type signature accordingly.




\section{Evaluation}\label{evaluation}
here is a big table that takes up a whole page

\subsection{Optimizations}

Since the standard library can be considered a relatively stable set of code, we can cache the refinement type inference to reduce the build time.
On my machine, it removes ~40 seconds from the build time.

\section{Related Work and Conclusions}\label{conclusions}
\section{Conclusions}
\label{conclusions}

Even for novice users, basic programming skills are becoming more commonplace, and synthesis tools must be able to provide users access to code.

\bibliographystyle{abbrvnat}
\bibliography{myBib}

\end{document}
