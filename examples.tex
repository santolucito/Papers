\section{Motivating Examples} 
\label{examples}

\subsection{Synthesis with the standard library}

As an introduction to \ourTool/, imagine that a user wants to synthesis the simple \codeinline{stutter} \cite{Osera:2015} function which duplicates each element of a list.
While in MYTH~\cite{Osera:2015} the focus was on lists as inductively defined data types, we are focusing on using the built in representation of a list in Haskell.
Haskell provides two relevant functions for this task in the standard library; \codeinline{concatMap :: (a -> [b]) -> [a] -> [b]} which applies a function over a list and concatenates the result, and \codeinline{replicate n :: Int -> a -> [a]} which will replicate an item n times into a list.
When working over the inductively defined list data type, MYTH requires three examples to find an appropriate function. Using \ourTool/, the user only needs to provides a single example to synthesize the program \codeinline{concatMap (replicate 2)} that fits that example on Haskell's builtin list type.

\begin{lstlisting}
exs :: [[Int] :-> [Int]]
exs = [[1, 2, 3] :-> [1, 1, 2, 2, 3, 3]]
\end{lstlisting}

If, in addition, the user develops a function \codeinline{dupl} which duplicates an element (see below), then \ourTool/ will provide the solution \codeinline{concatMap dupl}, as well as the solution listed above.

\begin{lstlisting}
dupl :: a -> [a]
dupl x = [x,x]
\end{lstlisting}

In the next example the user would like to synthesize a function that takes numbers from a list as long as the numbers are odd.
Again, only a single example is needed for \ourTool/ to unambiguously find  simple solution program, \codeinline{takeWhile odd}, using functions the from the standard library; \codeinline{takeWhile :: (a -> Bool) -> [a] -> [a]} to recurse over a list and take items until the predicate is false and \codeinline{odd :: Int -> Bool}. 

\begin{lstlisting}
exs :: [[Int] :-> [Int]]
exs = [[1, 2, 3] :-> [1]]
\end{lstlisting}

Another correct solution for this example might be \codeinline{head} to take the first element. Searching for first order functions is an active research direction \cite{DBLP:conf/aaip/Katayama09, flashFillPOPL}, but in this work we are focused only on higher order functions. This is part because the goal of natural synthesis is to provide useful, nontrivial functions to users. A discussion of integrating our tool with first order searching techniques in provided in Section \ref{evaluation}.

\subsection{Synthesis with user defined values}

Working on a set of user defined code is also a critical task \ourTool/ supports. In the next example the user has provided a binary tree data structure and a higher order function to map over it. We show the synthesis of the exceedingly (for the sake of brevity) simple program \codeinline{mapBTree not}. Doing such synthesis requires automatic reasoning about not only the user defined polymorphic data type, but also the higher order function they have defined over it.

\begin{lstlisting}
data BTree a = Nil |
               Branch (BTree a) a (BTree a)

mapBTree :: (a -> a) -> BTree a -> BTree 
mapBTree f Nil = Nil
mapBTree f (Branch b1 v b2) = 
  Branch (mapBTree f b1) (f v) (mapBTree f b2)

exs :: [BTree Bool :-> BTree Bool]
exs = [Branch Nil True Nil :->
       Branch Nil False Nil]
\end{lstlisting}

It may seem that if a user can write a higher order functions over custom data structures, they would not have a need to synthesize such functions.
However, imagine the incredibly common case of a user importing libraries.
Haskell's module system and large repository of libraries like Hackage and Stackage are an indispensable part of the language\cite{hackage,stackage}.
Often, a user is importing a library that is large, unfamiliar, and/or poorly documented.
Using \ourTool/, the user no longer needs an intimate knowledge of the library to makes use of the functions and datatypes, and can instead synthesize functions from examples.


\subsection{Synthesis with a DSL}

As an example, we show code to transpose a music value from the Euterpea DSL (domain specific library) for music\cite{euterpea}.
Among other things, Euterpea defines a tree-like datatype called \codeinline{Music} and various functions for manipulating these types.
The user only needs to express the basic datatype as examples, and \ourTool/ can synthesize the \codeinline{solution} program.
The solution utilizes the functions from Euterpea; \codeinline{mMap} for mapping over music values, and \codeinline{(trans::Int -> Music Pitch->Music Pitch)} to transpose a Music Pitch by a value.
This again requires automated reasoning about the properties of the library's data types and higher order functions.
Because we have synthesized a natural looking program, the user does not need to understand details of the library's function and data structures to be able to immediately gain an intuition about how the solution program works.

\begin{lstlisting}
import Euterpea

exs :: [Music Pitch :-> Music Pitch]
exs = [
  (Prim (Note qn (C,4)):+:Prim (Note qn (D,4)) :->
  (Prim (Note qn (D,4)):+:Prim (Note qn (E,4)) ]
        
solution = mMap (trans 2)
\end{lstlisting}

