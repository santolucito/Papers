As an introduction to \ourTool/, imagine that a user wants to synthesis the simple \codeinline{stutter} function that will duplicate each element of a list.
The user will provide an example, and \ourTool/ will synthesis a program \codeinline{concatMap (replicate 2)} that fit that example.

\begin{lstlisting}
exs :: [[Int] :-> [Int]]
exs = [[1, 2, 3] :-> [1, 1, 2, 2, 3, 3]]
\end{lstlisting}

We might also imagine that the user was able to complete a piece of this program by writing a function \codeinline{dupl} to duplicate an element. Now \ourTool/ will provide the solution \codeinline{concatMap dupl}, as well as the solution above.

\begin{lstlisting}
dupl :: a -> [a]
dupl x = [x,x]
\end{lstlisting}

Because \ourTool/ only searches for natural and idiomatic programs that use higher order functions, very few examples are needed. In this case the user wants a function that will take numbers from a list as long as the numbers are odd. Only a single example is needed for \ourTool/ to unambiguously find the function \codeinline{takeWhile odd}. Another valid function might be \codeinline{head} to take the first element. Searching for first order functions is an active research direction, but in this work we instead focus only on higher order functions. A discussion of integrating our tool with first order searching techniques in provided in Section \ref{evaluation}.

\begin{lstlisting}
exs :: [[Int] :-> [Int]]
exs = [[1, 2, 3] :-> [1]]
\end{lstlisting}

Working on user defined datatypes is also commonplace

\begin{lstlisting}

\end{lstlisting}

Haskell's module system and large repository of libraries like Hackage and Stackage are indispensable to a user\cite{hackage,stackage}. If a user is importing a library, \ourTool/ can also synthesize programs that use those functions and datatypes. As an example, we show code to transpose a music value from the Euterpea DSL for music\cite{Euterpea}. Among other things, Euterpea defines a tree-like datatype called \codeinline{Music} and various functions for manipulating these types. When given the examples below, \ourTool/ will synthesize the \codeinline{solution} program. The solution utilizes the functions from Euterpea; \codeinline{mMap} for mapping over music values, and \codeinline{(trans::Int->Music Pitch->Music Pitch)} to transpose a Music Pitch by a value. Without going into musical details, this is indeed a correct and sensible function. 

\begin{lstlisting}
import Euterpea

exs :: [Music Pitch :-> Music Pitch]
exs = [
  (Prim (Note qn (C,4)):+:Prim (Note qn (D,4)) :->
  (Prim (Note qn (D,4)):+:Prim (Note qn (E,4)) ]
        
solution = mMap (trans 2)
\end{lstlisting}
